\selectlanguage{english}
\begin{abstract}
Le dipendenze tra dati rappresentano uno dei metadati chiave per caratterizzare e profilare sorgenti multimediali e di big data. Tuttavia, rispetto ai database tradizionali, in questi nuovi contesti \`{e} stato necessario introdurre alcune approssimazioni nella definizione delle dipendenze e ideare algoritmi di scoperta per estrarli automaticamente dai dati. Ci\`{o} nonostante, le approssimazioni producono una proliferazione di dipendenze prodotte dagli algoritmi di rilevamento, il che rende difficile per un utente analizzarle efficacemente. A tal fine, presentiamo un tool per classificare e visualizzare le dipendenze scoperte tra big data e dati multimediali, attraverso un grafico dinamico immediato.
%\\[1cm]
\end{abstract}
\selectlanguage{italian}