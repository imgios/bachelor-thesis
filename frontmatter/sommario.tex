\selectlanguage{english}
\begin{abstract}
Le dipendenze tra dati rappresentano uno dei metadati chiave per caratterizzare e profilare sorgenti multimediali o in generale big data. Tuttavia, rispetto ai database tradizionali, in questi nuovi contesti \`{e} stato necessario introdurre alcune approssimazioni nella definizione delle dipendenze e ideare algoritmi di scoperta per estrarli automaticamente dai dati. Per il rilassamento delle dipendenze si mira ad effettuare un confronto approssimato ed a catturare dipendenze che non valgono sull'intero dataset. Ciononostante, le approssimazioni producono una proliferazione di dipendenze prodotte dagli algoritmi usati per il loro rilevamento, il che rende difficile per un utente analizzarle efficacemente. A tal fine, presentiamo un tool per classificare e visualizzare le dipendenze scoperte su big data, attraverso un grafico dinamico che permette di riassumere le dipendenze valide su un'istanza di database. In particolare, esso rappresenta la correlazione presente tra gli attributi, e come quest'ultima evolve al variare delle cardinalit\`{a} del lato sinistro delle dipendenze e delle soglie di confronto approssimato sia sul lato destro che sul lato sinistro delle dipendenze.
%\\[1cm]
\end{abstract}
\selectlanguage{italian}