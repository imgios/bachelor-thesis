\chapter{Stato dell'arte}%\label{1cap:spinta_laterale}
\label{cap2:stato_arte}
% [titolo ridotto se non ci dovesse stare] {titolo completo}
%

%\begin{citazione}
Questo capitolo illustra lo stato dell'arte e i lavori presenti in letteratura sugli aspetti di ricerca trattati nel nostro studio. Il problema della visualizzazione di grandi insiemi di \acrshort{rfds} automaticamente estratte dai dati \`{e} un problema recente. Ci\`{o} \`{e} dovuto ai recenti miglioramenti delle prestazioni degli algoritmi per l'estrazione delle \acrshort{rfds} dai dati e al potenziale delle soluzioni parallele che consentono la loro scoperta dai big data. Quindi, \`{e} sorto solo di recente il problema della gestione di grandi insiemi di \acrshort{rfds}, molti dei quali differiscono solo per il valore delle soglie di approssimazione. Per questo motivo, in letteratura non vi sono soluzioni proposte per gestire tale complessit\`{a}, n\'{e} in termini di ranking e riepilogo delle \acrshort{rfds} trovate, n\'{e} in termini di tecniche di visualizzazione.\par
Gran parte del lavoro \`{e} stato svolto nell'ambito delle tecniche di visualizzazione dei dati e dei metadati \cite{topicmodeling}. Comunque, nel contesto di visualizzazione dei metadati \`{e} stata posta poca attenzione alla visualizzazione delle dipendenze. Un tentativo in questa direzione \`{e} stato fatto in \cite{frameworktoanalyzeinfo}, dove oltre a fornire una tecnica per visualizzare le \acrshort{fds}, il concetto stesso di \acrshort{fd} viene sfruttato per visualizzare altre caratteristiche dei dati. In particolare, gli autori definiscono lo schema di visualizzazione funzionale che rappresenta una mappatura visiva come un insieme di \acrshort{fds} tra dati e attributi visivi.\par
Un'altra proposta interessante \`{e} il progetto Metanome \cite{dataprofilingwithmetanome}, che \`{e} una piattaforma che include diversi algoritmi per la ricerca automatica di metadati complessi, includendo dipendenze funzionali e di inclusione. Per facilitarne l'analisi, Metanome offre diverse metriche di ranking, insieme a tecniche di visualizzazione per \acrshort{fds} e dipendenze di inclusione.\par
Tra gli approcci correlati alla visualizzazione delle dipendenze, vale la pena menzionare quelli che puntano alla visualizzazione delle regole d'associazione (AR) \cite{chenvisualanalysis,visualassrules,wifisviz,assocexplorer}. Infatti, il concetto di AR \`{e} in qualche modo correlato a quello di \acrshort{rfd}. In particolare, sebbene una AR rappresenti co-occorrenze di valori piuttosto che una correlazione tra attributi, il formalismo usato per rappresentare le \acrshort{rfds} e le AR \`{e} simile e possono essere visualizzate utilizzando una metafora comune. Inoltre, le AR incorporano il concetto di confidence che \`{e} altamente correlato al concetto di extent delle \acrshort{rfds}. Tra i metodi di visualizzazione per rappresentare le ARs vale la pena menzionare il mosaic plot \cite{visualassrules}, lo scatter plot \cite{assocexplorer}, il node-link graph \cite{wifisviz} ed il matrix \cite{vaet}. Il tool ARVis fu proposto per validare ed esplorare ARs, mentre il tool in \cite{visualassrulesusingmatrix} fornisce pi\`{u} interfacce per ispezionare visualmente l'insieme generale di AR. Infine, \`{e} stata presentata una tecnica di visualizzazione basata su matrice gerarchica in \cite{chenvisualanalysis}, dove gli autori forniscono un'organizzazione gerarchica delle loro matrici per migliorare l'esplorazione e la visualizzazione delle AR.\par
Come accennato in precedenza, l'analisi delle \acrshort{rfds} \`{e} molto pi\`{u} complessa, poich\'{e} rispetto alle \acrshort{fds}, gli algoritmi di ricerca per le \acrshort{rfds} potrebbero produrre in output molte pi\`{u} dipendenze, molte delle quali potrebbero avere lievi differenze nelle soglie che rappresentano il grado si approssimazione. Questo aumenta enormemente la necessit\`{a} di escogitare metodi di visualizzazione e di ranking per permettere all'utente un'analisi rapida dei risultati dei processi di ricerca delle \acrshort{rfds} e migliorare i processi decisionali basati su di essa. Date le caratteristiche differenti delle \acrshort{rfds} rispetto alle \acrshort{fds}, le tecniche di visualizzazione e di ranking attualmente disponibili per le \acrshort{fds} non sono adatte ai nostri obiettivi, n\'{e} sono facilmente adattabili per visualizzare aspetti peculiari delle \acrshort{rfds}, come le soglie di similudine e di coverage, che definiscono il grado di approssimazione (rilassamento) applicabile.
%\end{citazione}

\newpage