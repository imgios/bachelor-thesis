\phantomsection
%\addcontentsline{toc}{chapter}{Introduzione}
\chapter{Introduzione}
\markboth{Introduzione}{}
% [titolo ridotto se non ci dovesse stare] {titolo completo}

\section{Motivazioni e Obiettivi} %\label{1sec:scopo}
L'evoluzione da dati tradizionali a dati complessi (e.g., multimediali, geografici e fuzzy) e in generale ai big data, ha sollevato la necessit\`{a} di progettare metodi e strumenti per estrarre automaticamente e visualizzare le informazioni e propriet\`{a} che sussistono tra questi. Tali propriet\`{a} includono le classificazioni, le regole di associazione e diversi tipi di metadata \cite{profiling-relational-data}, come i value pattern, le foreign key e le dipendenze dei dati, che possono essere di tipo multivalore, inclusione e \acrlong{fds} (\acrshort{fds}) \cite{surveydatabasedependency}. Queste ultime venivano usate principalmente nei tradizionali database alfanumerici a scopi di normalizzazione\footnote{La \textbf{normalizzazione} \`{e} un processo atto ad eliminare le ridondanze ed il rischio di incoerenza dal database.}. Successivamente, con l'evoluzione dei domini d'applicazione, che hanno portato sempre pi\`{u} alla necessit\`{a} di eseguire operazioni complesse sui database, nuovi tipi di \acrlong{fds} sono state definite, come ad esempio le Dipendenze Funzionali Fuzzy \cite{ffdandlljoin} e le Dipendenze Multimediali \cite{nomalizationframework}, le quali introducono alcune approssimazioni necessarie per confrontare dati complessi, ma anche per tener conto dei confronti testuali, dei valori mancanti, delle dipendenze parzialmente valide ed altre approssimazioni nel contesto di applicazioni che sfruttano big data. A tal fine, sono state introdotte oltre trenta definizioni differenti di Dipendenze Approssimate, chiamate anche \acrlong{rfds} (\acrshort{rfds}) \cite{rfdsurvey}. Nel contesto dei big data, le \acrlong{rfds} insieme ad altri metadati di profilazione sono utilizzate per diversi scopi, tra cui pulizia dei dati, ottimizzazione delle query e cos\`{i} via. Tuttavia, in questo contesto potrebbe essere difficile specificare le \acrlong{rfds} in fase di progettazione dei database, in quanto oltre alla necessit\`{a} di specificare gli attributi coinvolti in una dipendenza, \`{e} necessario definire anche i loro livello di approssimazione, come ad esempio la similitudine tra valori di attributi e le soglie per indicare la porzione di dataset su cui una \acrlong{rfd} vale. Quindi, \`{e} stato necessario progettare metodi per la ricerca automatica delle \acrlong{rfds} a partire dai dati \cite{ddiscoveryfromdata}. La ricerca delle \acrlong{rfds} dai dati \`{e} possibile grazie alla disponibilit\`{a} di raccolte di big data, poich\'{e} forniscono un'elevata rilevanza statistica per dipendenze approssimate che altrimenti dovrebbero essere specificate in fase di progettazione, in base alla semantica dei dati. D'altro canto, sorgono due problemi principali con gli algoritmi di scoperta di \acrlong{rfds} applicati ai big data:
\begin{enumerate}
    \item La complessit\`{a} computazionale,
    \item La visualizzazione dei risultati.
\end{enumerate}
Il primo \`{e} un problema gi\`{a} sufficientemente complesso anche con le \acrlong{fds} tradizionali e la sua complessit\`{a} cresce considerabilmente con l'introduzione delle approssimazioni, specialmente quando il numero di attributi cresce. Il secondo problema riguarda la complessit\`{a} dei risultati, perch\'{e} gli algortimi di ricerca delle \acrlong{rfds} spesso restituiscono in output un enorme set di dipendenze, con molte combinazioni di soglie differenti, il che rende difficile per un utente ricavare informazioni utili da queste. Ad oggi non \`{e} stato fornito alcun metodo di visualizzazione per le \acrlong{rfds}. L'obiettivo, quindi, \`{e} quello di progettare uno strumento che faciliti l'individuazione e la lettura delle informazioni utili riguardanti le \acrlong{rfds}. La tecnica di visualizzazione da utilizzare si applica alle \acrlong{rfds} scoperte da big data e rappresenta sia le dipendenze scovate sia le loro soglie, includendo una strategia di classificazione che mira a selezionare le \acrlong{rfds} pi\`{u} rilevanti e significative su un determinato dataset.

\section{Struttura della tesi}
La tesi \`{e} strutturata come segue. Il capitolo \ref{cap2:stato_arte} esamina gli approcci esistenti in letteratura per visualizzare i metadati estratti dai big data. Il capitolo \ref{cap3:fd} fornisce alcune definizioni di background sulle \acrlong{rfds} ed una breve introduzione sugli algoritmi per la loro estrazione dai dati. Il capitolo \ref{cap4:visual_representation} introduce il concetto di minimalit\`{a} per filtrare le \acrlong{rfds} rilevanti ed una metafora di visualizzazione per rappresentare tali dipendenze. Il capitolo \ref{cap5:dependensee} introduce \textit{Dependensee}, il tool sviluppato durante il lavoro di tesi per la visualizzazione di insiemi minimali di \acrlong{rfds}.