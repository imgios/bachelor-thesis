\chapter{Conclusioni} %\label{1cap:spinta_laterale}
% [titolo ridotto se non ci dovesse stare] {titolo completo}
%
In questo capitolo verranno esposte le conclusioni relative al lavoro di tesi effettuato ed ai possibili sviluppi futuri.

\section{Conclusioni}
L'obiettivo di questa tesi \`{e} stato la realizzazione di un applicazione per la rappresentazione grafica di enormi insiemi di \acrlong{rfds} minimali, mediante l'utilizzo di una metafora di visualizzazione chiara ed immediata. L'applicazione deve interpretare un dataset di \acrshort{rfds} minimali e, successivamente, deve creare un grafico relativo al dataset analizzato. Quindi, la prima parte del lavoro di tesi si \`{e} esplorato il concetto di \acrlong{fd} e poi di \acrlong{rfd}, con un accenno agli algoritmi di ricerca delle \acrlong{rfds}. Tali informazioni sono state raccolte nel capitolo \ref{cap3:fd}. Dopo l'introduzione ai concetti appena citati, ci si \`{e} soffermati sulla rappresentazione grafica delle \acrlong{rfds}, introducendo il concetto di minimalit\`{a} delle \acrlong{fds} e \acrlong{rfds}, esplorando la metafora di visualizzazione per la rappresentazione delle dipendenze proposta in \cite{mdvisualization}. Tali informazioni, invece, sono state raccolte nel capitolo \ref{cap4:visual_representation}. La terza ed ultima parte della tesi, descritta nel capitolo \ref{cap5:dependensee}, \`{e} stata dedicata alla presentazione delle fasi di sviluppo vero e proprio dell'applicazione ed alla successiva valutazione sperimentale. Questa ultima fase non ha presentato particolari problemi sulla rappresentazione grafica degli insiemi minimali, poich\'{e} i grafici sono creati dinamicamente. Durante la verifica sperimentale si \`{e} avuto modo di testare il corretto funzionamento dell'applicazione con l'utilizzo di diversi dataset. L'esito \`{e} stato positivo, la rappresentazione risultante per ogni dataset si \`{e} dimostrata efficace per analizzare visivamente tali dataset, in modo semplice ed efficace, riducendo di molto la difficolt\`{a} all'utente che la interpreta.

\section{Sviluppi futuri}
Un possibile sviluppo del progetto potrebbe essere la realizzazione di un'interfaccia grafica pi\`{u} gradevole, fornendo molte pi\`{u} informazioni e linee guida generali per guidare l'utente nell'utilizzo dell'applicazione. Inoltre, potrebbero essere implementate ulteriori funzioni che possano fornire informazioni pi\`{u} dettagliate. Ci\`{o} sarebbe possibile, ad esempio, tramite l'implementazione di ulteriori metafore di visualizzazione descritte in \cite{mdvisualization}, le quali forniscono informazioni pi\`{u} dettagliate riguardo le \acrlong{rfds} minimali di un determinato sottoinsieme del dataset. Infine, potrebbero essere previste delle applicazioni mobile per Android ed iOS, visto il grande utilizzo sempre pi\`{u} crescente degli smartphone e dei tablet in ambito lavorativo. Questo ulteriore sviluppo \`{e} facilitato dall'utilizzo di Ionic, il quale si propone come un Software Development Kit (SDK) per lo sviluppo non solo di applicazioni web, ma anche di applicazioni mobile ibride. Infine, si potrebbe pensare di rendere l'interfaccia presentata dinamica, in modo che essa possa visualizzare l'evoluzione delle \acrshort{rfds} dopo l'inserimento di nuove tuple nel dataset. Per poter effettuare ci\`{o}, l'interfaccia dovrebbe essere messa in comunicazione con un algoritmo incrementale per la scoperta di \acrshort{rfds}.

\newpage